\documentclass{amsart}
\usepackage{amsmath}
\usepackage{amssymb}
\usepackage{dsfont}
\usepackage{multicol}
\usepackage{proof}
\usepackage{setspace}
\usepackage{stmaryrd}

\newtheorem{thm}{Theorem}[section]
\newtheorem{prop}[thm]{Proposition}
\newtheorem{lem}[thm]{Lemma}
\newtheorem{cor}[thm]{Corollary}
\theoremstyle{definition}
\newtheorem{definition}[thm]{Definition}
\newtheorem{example}[thm]{Example}

\theoremstyle{remark}
\newtheorem{remark}[thm]{Remark}

\numberwithin{equation}{section}

\newcommand{\N}{\mathbb{N}}
\newcommand{\Z}{\mathbb{Z}}
\newcommand{\formation}[1]{\text{$#1$-formation}}
\newcommand{\intro}[1]{\text{$#1$-intro}}
\newcommand{\elim}[1]{\text{$#1$-elim}}
\newcommand{\extensionality}[1]{\text{$#1$-ext}}
\newcommand{\qbox}[1]{\lbrack#1\rbrack}
\newcommand{\pair}[1]{\langle#1\rangle}
\newcommand{\singsym}{\mathbb{S}}
\newcommand{\sing}[2]{\singsym_{#1}(#2)}
\newcommand{\univ}{\mathbb{U}}
\newcommand{\void}{\mathbf{0}}
\newcommand{\unit}{\mathbf{1}}
\newcommand{\SET}{\textbf{Set}}
\newcommand{\sem}[1]{\left\llbracket#1\right\rrbracket}
\newcommand{\quo}[1]{\left\langle#1\right\rangle}
% \DeclareMathOperator{\dist}{dist} % The distance.
% \DeclareMathOperator{\dist}{dist} % The distance.

\begin{document}
\title{Theories with Judgement}
\author{Jonathan Sterling}
\address{AlephCloud Systems}
\email{jon@jonmsterling.com}

\maketitle

A \emph{Theory} is given by a language $\mathcal{L}$, an inductively defined set of judgements
$\mathcal{J}$, and an explanation of their meaning $\sem{-} :
\SET^\mathcal{J}$.  One such theory is the theory $\mathbf{Nat}$ of the natural
numbers, whose terms are the numerals:
\[
  \begin{array}{lcl}
    \mathcal{L}_\mathbf{Nat}
      &::= &\mathsf{zero}\\
      &\mid &\mathsf{succ}\;\quo{\mathcal{L}_\mathbf{Nat}}
  \end{array}
\]
The theory $\mathbf{Nat}$ has only a single form of judgement, which asserts that a term is a natural number.
\[
  \mathcal{J}_\mathbf{Nat} = \{n\;\mathrm{nat} \mid n : \mathcal{L}_\mathbf{Nat} \}
\]
The judgement is then interpreted over the syntax recursively:
\begin{align*}
  \sem{\mathsf{zero}\;\mathrm{nat}}_\mathbf{Nat} &= \top\\
  \sem{\mathsf{succ}\;n\;\mathrm{nat}}_\mathbf{Nat} &= \sem{n\;\mathrm{nat}}_\mathbf{Nat}
\end{align*}
Going forward, we'll equivalently present the judgements and their
interpretations in terms of ``canonical constructors'' in the ambient
metalanguage, as follows:
\begin{align*}
  (-\;\mathrm{nat}) &: \mathcal{J}_\mathbf{Nat}^{\mathcal{L}_\mathbf{Nat}}
\end{align*}
\begin{gather*}
  \infer{\mathsf{zero}\;\mathrm{nat}}{}
  \qquad
  \infer{\mathsf{succ}\;n\;\mathrm{nat}}{n\;\mathrm{nat}}
\end{gather*}
Now, this is not a particularly interesting theory, since its single form of
judgement is true at all instantiations, but it served to illustrate the
construction of a theory with judgement.

A more interesting theory is that of names, $\mathbf{Nm}$; we take
$\mathcal{L}_\mathbf{Nm}$ to be the countably infinite set of strings of
letters $\{\mathsf{a},\mathsf{b},\mathsf{c}\dots\}$. Note that we refer to a
canonical name in \textsf{sans serif} font, whereas we quantify schematically
over variables $x:\mathcal{L}_\mathbf{Nm}$ in \textit{italic} font. Then, the
judgements are given as follows:
\begin{align*}
  (-\mathop{\#}-) &: \mathcal{J}_\mathbf{Nm}^{\mathcal{L}_\mathbf{Nm}\times\mathcal{L}_\mathbf{Nm}}
  \tag{Apartness}
\end{align*}
We can take the interpretation of the apartness judgement as primitive.

Now we can define the theory of contexts of assumptions over some other theory
$\mathbf{T}$, which we will call $\mathbf{Ctx}[\mathbf{T}]$; we introduce the
following syntax and judgements:
\[
  \begin{array}{lcl}
    \mathcal{L}_{\mathbf{Ctx}[\mathbf{T}]}
      &::= &\cdot\\
      &\mid & \quo{\mathcal{L}_{\mathbf{Ctx}[\mathbf{T}]}},\quo{\mathcal{L}_\mathbf{Nm}}:\quo{\mathcal{J}_\mathbf{T}}
  \end{array}
\]

\begin{align*}
  (-\;\mathrm{ctx}) &: \mathcal{J}_{\mathbf{Ctx}[\mathbf{T}]}^{\mathcal{L}_{\mathbf{Ctx}[\mathbf{T}]}}\\
  (-\notin-) &: \mathcal{J}_{\mathbf{Ctx}[\mathbf{T}]}^{\mathcal{L}_\mathbf{Nm}\times\mathcal{L}_{\mathbf{Ctx}[\mathbf{T}]}}\\
  (-\ni-:-) &: \mathcal{J}_{\mathbf{Ctx}[\mathbf{T}]}^{\mathcal{L}_{\mathbf{Ctx}[\mathbf{T}]}\times\mathcal{L}_\mathbf{Nm}\times\mathcal{J}_\mathbf{T}}\\
  (-\leq-) &: \mathcal{J}_{\mathbf{Ctx}[\mathbf{T}]}^{\mathcal{L}_{\mathbf{Ctx}[\mathbf{T}]}\times\mathcal{L}_{\mathbf{Ctx}[\mathbf{T}]}}
\end{align*}

And the meanings of these judgements are given inductive-recursively in terms of the
following rules:
\begin{gather*}
  \infer{\cdot\;\mathrm{ctx}}{}
  \qquad
  \infer{
    \Gamma,x:J\;\mathrm{ctx}
  }{
    \Gamma\;\mathrm{ctx}
    & x\notin\Gamma
  }\\[6pt]
  \infer{x\notin\cdot}{}
  \qquad
  \infer{
    x\notin(\Gamma,y:J)
  }{
    x\notin\Gamma
    & x\mathop{\#}y
  }\\[6pt]
  \infer{\Gamma,x:J\ni x:J}{}
  \qquad
  \infer{
    \Gamma,x:J\ni y:J'
  }{
    \Gamma\ni y:J'
  }\\[6pt]
  \infer{\cdot\leq\Gamma}{}
  \qquad
  \infer{
    \Delta,x:J\leq\Gamma
  }{
    \Delta\leq\Gamma &
    \Gamma\ni x:J
  }
\end{gather*}

Finally, we can iterate the process in order to get a theory $\mathbf{IPC}$ of
intuitionistic propositional logic, which introduces a notion of hypothetical
judgement: but note that we did not need to include that as part of the general
framework, since we have defined it internally in the theory of contexts above.
\[
  \begin{array}{lcl}
    \mathcal{L}_\mathbf{IPC}
      &::= &\bot\\
      &\mid &\top\\
      &\mid &\quo{\mathcal{L}_\mathbf{IPC}}\land\quo{\mathcal{L}_\mathbf{IPC}}\\
      &\mid &\quo{\mathcal{L}_\mathbf{IPC}}\lor\quo{\mathcal{L}_\mathbf{IPC}}\\
      &\mid &\quo{\mathcal{L}_\mathbf{IPC}}\supset\quo{\mathcal{L}_\mathbf{IPC}}
  \end{array}
\]
The theory has a single form of judgement,
\[
  -\;\mathrm{true}\;[-] : \mathcal{J}_\mathbf{IPC}^{\mathcal{L}_\mathbf{IPC}\times\mathcal{L}_{\mathbf{Ctx}[\mathbf{IPC}]}}
\]
The definition of the theory $\mathbf{IPC}$ is circular, but not viciously so;
it is inductive-recursive. Then, the meaning of this judgement is given by the
following rules:
\begin{gather*}
  \infer{\top\;\mathrm{true}\;[\Gamma]}{}
  \qquad
  \infer{P\;\mathrm{true}\;[\Gamma]}{\bot\;\mathrm{true}\;[\Gamma]}\\[6pt]
  \infer{
    P\land Q\;\mathrm{true}\;[\Gamma]
  }{
    P\;\mathrm{true}\;[\Gamma] &
    Q\;\mathrm{true}\;[\Gamma]
  }
  \qquad
  \infer{
    R\;\mathrm{true}\;[\Gamma]
  }{
    P\land Q\;\mathrm{true}\;[\Gamma] &
    R\;\mathrm{true}\;[\Gamma,x:P\;\mathrm{true}\;[\Gamma];y:Q\;\mathrm{true}\;[\Gamma,x:P\;\mathrm{true}\;[\Gamma]]]
  }\\[6pt]
  \infer{
    P\;\mathrm{true}\;[\Gamma]
  }{
    P\lor Q\;\mathrm{true}\;[\Gamma]
  }
  \qquad
  \infer{
    Q\;\mathrm{true}\;[\Gamma]
  }{
    P\lor Q\;\mathrm{true}\;[\Gamma]
  }
  \qquad
  \infer{
    R\;\mathrm{true}\;[\Gamma]
  }{
    P\lor Q\;\mathrm{true}\;[\Gamma] &
    R\;\mathrm{true}\;[\Gamma,x:P\;\mathrm{true}\;[\Gamma]] &
    R\;\mathrm{true}\;[\Gamma,x:Q\;\mathrm{true}\;[\Gamma]]
  }\\[6pt]
  \infer{
    P\supset Q\;\mathrm{true}\;[\Gamma]
  }{
    Q\;\mathrm{true}\;[\Gamma,x:P\;\mathrm{true}\;[\Gamma]]
  }
  \qquad
  \infer{
    Q\;\mathrm{true}\;[\Gamma]
  }{
    P\supset Q\;\mathrm{true}\;[\Gamma] &
    P\;\mathrm{true}\;[\Gamma]
  }\\[6pt]
  \infer{
    P\;\mathrm{true}\;[\Gamma]
  }{
    \Gamma\ni x:P\;\mathrm{true}\;[\Delta] &
    \Delta\leq\Gamma
  }
\end{gather*}

\end{document}

